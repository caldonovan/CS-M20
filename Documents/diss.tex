% ! THIS MUST BE COMPILED USING XeLaTeX !

\documentclass{article}
\usepackage{listings}
\usepackage[utf8]{inputenc}
\usepackage{fancyhdr}
\usepackage{xcolor}
\usepackage{caption}
\usepackage{graphicx}
\usepackage{amsmath}
\usepackage{booktabs}
\usepackage{textcomp}
\usepackage{rotating}
\usepackage{fontspec}
\usepackage{libertine}
\usepackage{lipsum}
\usepackage[T1]{fontenc}
\usepackage[english]{babel}

% Layout related commands.
\setmainfont{Linux Libertine O}
\graphicspath{ {./images/} }
\lstdefinestyle{code}{basicstyle=\ttfamily\footnotesize, breakatwhitespace=false,
    breaklines=true, keywordstyle=\color{magenta}, commentstyle=\color{codegreen},
    keepspaces=true, showspaces=false, showstringspaces=false}
\lstset{style=code}
\pagestyle{fancy}
\fancyhf{}

\rhead{Callum Donovan}
\lhead{Implementing SAT Algorithms in Software}
\rfoot{Page \thepage}
\lfoot{CS-M20 - 1915769}

\title{\bfseries Implementing SAT Algorithms in Software}
\author{Callum Donovan}
\date{September 2020}

\setlength{\parindent}{4em}
\setlength{\parskip}{0.5em}

\begin{document}

\begin{titlepage}
    \begin{center}
        \Large{\bfseries Implementing SAT Algorithms in Software} \\
        \vspace{4cm}
        \begin{center}
            \includegraphics[scale=0.2]{swan.jpg}
        \end{center}
        \vspace*{\fill}
        \bfseries{\large Callum Donovan \\
            1915769 \\
            September 1, 2020 \\
            Swansea University \\}
    \end{center}
\end{titlepage}

\section{Abstract}
% TODO: Use the word "implement" a lot less!
The Boolean Satisfiability Problem is a fundimental part of Computer Science. First proven to be an
NP-Complete problem by both Stephen Cook in 1971, and Leonid Levin in 1973[!]. Since then, many more
NP-Complete problems have been identified, along with their respective uses in the real world
outside of pure computer science. Due to this, there has been a growing need for solvers that can
effectively and efficiently process these problems. In the last two decades alone, huge progress has
been made to coincide with the advancements in technology. And more SAT solvers have appeared that
can be deployed in industries where they are most needed. 

This paper explores the basic concept behind SAT solvers, and how they can be implemented in
software using modern programming languages and tools. The algorithm that will be
implemented is one proposed by Donald Knuth in his book "The Art of Computer Programming". Knuth proposes
many algorithms, ranging from the most basic backtracking based algorithm, to an advanced
implementation of WalkSAT[!]. We will explore the fundimentals of how to implement these algorithms,
implement one of them, then test its performance against a suite of SAT problems.

The result of this paper will be a SAT solver implemented in C++, using the data structures and
steps provided by Donald Knuth in his book "The Art of Computer Programming".

\newpage
\tableofcontents

\newpage
\section{Introduction}
\lipsum[2-4]
% TODO: Basically talk about the project and the motivation behind it

\newpage
\section{Background}
% TODO: Talk about the background research we carried out from CSCM10
Before we get into the implementation and software, it is always good to have a solid understanding
of the topic at hand. SAT in particular is quite a deep topic, it is easy to spend a long time
researching into all the intricacies[!]. For this paper, we will just be concentrating on the
general concepts and ideas behind 3SAT problems. 

\subsection{SAT Fundimentals}
% TODO: Explain the fundimentals of SAT and the people who found it.
\lipsum[2-4]

\subsection{3SAT}
\lipsum[2-4]

\subsection{Modern SAT}
\lipsum[2-4]

\newpage
\section{Related Work}
% TODO: Add related work, such as alternative FOSS solutions.
Moving on from the background of SAT, we can look into existing solutions that are freely available.
As of 2020, there are plenty of FOSS programs that can be used to solve large sets of problems. This
is in part due to the regularity of the SAT competitions that encourage people to build
groundbreaking solvers. 

A popular and early solver that has won many competitions[!] is Minisat. Written in C++ and boasting
just a mere few hundred lines of code, Minisat has been proven to be an effective tool for medium
sized problem sets. First written in 2003 by Niklas Een[!] and Niklas Sorensson[!], its primary goal
was to help developers and researchers get into the field of SAT solving by providing a simple
interface and minimal codebase. 

\newpage
\section{Design}
\lipsum[2-4]
% TODO: Talk about how we've design the code based off of the algorithm and preceeding content in the book.

\newpage
\section{Implementation}
\lipsum[2-4]
% TODO: Talk about how we are going to implement this. Not so much about the actual implementation itself!

\newpage
\section{Testing}
\lipsum[2-4]
% TODO: Talk about the testing of the implemented code and whether it meets the requirements that we have set previously

\newpage
\section{Evaluation}
\lipsum[2-4]
% TODO: Talk about the challenges we faced and also what we could improve if doing the project again

\newpage
\section{Conclusion}
\lipsum[2-4]
% TODO: Conclude on our findings and all the preceeding work we have done.

\end{document}
